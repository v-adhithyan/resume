\documentclass[11pt,a4paper]{article}
\usepackage{geometry}
\usepackage{titlesec}
\usepackage{ragged2e}
\usepackage[hidelinks]{hyperref}
\usepackage{tabto}
\usepackage{enumitem}
\usepackage{multicol}
\setlength{\parskip}{0em}
\hypersetup{
	colorlinks=false
}
\urlstyle{same}
%\titlespacing{\section}{0pt}{0em}{0pt}{}
\geometry{left=20mm,right=20mm,top=20mm,bottom=20mm}
\begin{document}
\begin{center}
{\LARGE{\textbf{ADHITHYAN V}}}\\
{Ph:9498090493\hfill}{\hfill v.adhithyan@gmail.com} \\
\emph{Site} \href{https://v-adhithyan.github.io}{https://v-adhithyan.github.io} \\
\emph{Github} \href{https://github.com/v-adhithyan}{https://github.com/v-adhithyan}\\
\noindent\makebox[\linewidth]{\rule{\paperwidth}{0.4pt}}
\end{center}
\begin{multicols}{2}
\section*{VISION}
\par To contribute to the profitability and sustainability of organization with my expertise.
\section*{MISSION}
\par Seeking a challenging role in an organization where I can contribute to the growth of organization as well as develop my skills.
\end{multicols}
\section*{EXPERIENCE}
\emph{Business consultant Intern, Launchpad LLC}{\hfill May 2017 -  June 2017}
\begin{itemize}[noitemsep]
\item Performed a gap analysis of e-governance solutions for Chennai with respect to European smart cities.
\item Validated gap analysis using market research.
\item Prepared proposal based on the above work following the Standard Operating Procedures (SOP).
\end{itemize}
\emph{Member Technical Staff, Zoho}{\hfill June 2015 - May 2016}
\begin{itemize}[noitemsep]
\item Worked in Zoho Social and Zoho CRM.
\item In Zoho CRM, I monitored Cassandra for a feature called Sales Inbox with a colleague.
\item In Zoho Social, I fixed bugs, optimized code base, coded REST API's and proposed a new feature. \emph {Technologies used: Java, MySQL, Redis, Cassandra}
\end{itemize}
\emph{Intern, Zoho}{\hfill December 2014 - February 2015}
\begin{itemize}[noitemsep]
\item Studied code base of Zoho Social and became familiar with various internal frameworks, IDE used for development and testing in Zoho.
\end{itemize}
\section*{EDUCATION}
\textbf{Master of Business Administration (HR, Finance) {\hfill April 2018 (Expected)}}\\
College of Engineering, Guindy - Anna University {\hfill \textbf{CGPA: 8.86/10} (till semester 2)}\\
\textbf{BE., Computer Science and Engineering {\hfill April 2015}}\\
College of Engineering, Guindy - Anna University {\hfill \textbf{CGPA: 8.04/10}}\\
\section*{SKILLS}
\TabPositions{3cm}
\begin{itemize}[noitemsep]
\item Languages \tab Java, Python, C, C++, golang, Javascript, node.js, PHP
\item Databases \tab MySQL, MongoDB, Cassandra, Redis
\item IDE, SDK \tab Netbeans, Eclipse, Android SDK, Git
\item Testing tools \tab Selenium
\item Others \tab Microsoft office
\end{itemize}
\section*{PROJECTS}
\emph{Drona}{\hfill May 2017}
\begin{itemize}[noitemsep]
\item Android app that displays management related news from hand curated list of various management sites. \emph{Technology used: Kotlin}
\item \emph{Github} : \href{https://github.com/v-adhithyan/drona}{https://github.com/v-adhithyan/drona}
\item Available in \href{https://play.google.com/store/apps/details?id=ceg.avtechlabs.mba}{\emph{Playstore}}.
\end{itemize}
\emph{itunes-controller}{\hfill August 2016}
\begin{itemize}[noitemsep]
\item A macOS background utility that recognizes person in front of system and controls itunes playback. \emph{Built using
Python, OpenCV, Applescript.}
\item It is available as a pip package.
\item \emph{Github} : \href{https://github.com/v-adhithyan/itunes-controller}{https://github.com/v-adhithyan/itunes-controller}
\end{itemize}
\emph{PrettySeekBar}{\hfill June 2016}
\begin{itemize}[noitemsep]
\item Circular Seekbar library for Android inspired from analog clock. \emph{Technology used: Java.}
\item \emph{Github} : \href{https://github.com/v-adhithyan/PrettySeekBar}{https://github.com/v-adhithyan/}
\end{itemize}
\emph{AutoLogout}{\hfill February - March 2016}
\begin{itemize}[noitemsep]
\item A chrome extension that logs out from popular social networking sites after 10 minutes. \emph{Built using javascript.}
\item Available in \href{https://chrome.google.com/webstore/detail/auto-logout/affkccgnaoeohjnojjnpdalhpjhdiebh?hl=en}{\emph{Chrome web store}}.
\item \emph{Github} : \href{https://github.com/v-adhithyan/AutoLogout}{https://github.com/v-adhithyan/AutoLogout}
\end{itemize}
\emph{Cricket Score Applet}{\hfill March 2016}
\begin{itemize}[noitemsep]
\item A macOS menulet that displays live cricket score and news. \emph{Built using: swift.}
\item \emph{Github} : \href{https://github.com/v-adhithyan/cricket-score-applet}{https://github.com/v-adhithyan/cricket-score-applet}
\end{itemize}
\emph{Research paper trending topic finder}{\hfill December 2014 - April 2015}
\begin{itemize}[noitemsep]
\item  IEEE-KDE dataset was used to find the trending areas of research for years 2012, 2013 and 2014. \emph{Technologies
used: Java, Hadoop, MySQL.}
\end{itemize}
\emph{Comprehension answering system}{\hfill March 2014}
\begin{itemize}[noitemsep]
\item Developed as a part of Compilers lab. \emph{Technologies used: flex, yacc.}
\item When a passage and a question is given, the system will display relevant answer from the passage.
\end{itemize}
\emph{Offmail}{\hfill March 2014}
\begin{itemize}[noitemsep]
\item Developed as a part of FOSS lab. \emph{Technologies used: PHP, MySQL.}
\item The main functionality of system is the composed mail will be saved when the system is offline and it will be sent when internet
becomes available.
\end{itemize}
%\emph{Intelligent contact finder}{\hfill March 2014}
%\begin{itemize}[noitemsep]
%\item Developed as a part of Mobile lab. \emph{Technologies used: Android, Java, SQLite.}
%\item Android app that list contacts based on descending frequencies of calls made when searched.
%\end{itemize}
\emph{TextEditor}{\hfill March 2013}
\begin{itemize}[noitemsep]
\item Developed as a part of Java lab. \emph{Built using: Java}.
\item A simple text editor which can be used to open and save text files. Additionally users can find and replace
words.
\end{itemize}
\section* {PUBLICATIONS}

\begin{itemize}[noitemsep]
\item \href{http://www.ijstm.com/images/short_pdf/1427308779_1228.pdf}{IJSTM - A Novel Ontology based Framework for mapping Research Paper Titles to topic domains}
\item \href{https://www.ijser.org/researchpaper/A-Novel-Ontology-based-Framework-for-Trend-Detection-from-Research-Paper-Titles.pdf}{IJSER - A Novel Ontology based Framework for Trend Detection from Research Paper Titles}
\end{itemize}

\section*{MISC}
\begin{itemize}[noitemsep]
\item Completed online machine learning course offered by Andrew NG in coursera.org with 93.2\%.
\item Organized debugging event of aBACUS 2015, an inter college symposium conducted by CEG.
\item Secured educational district 3rd place in 12th Board Exams (2010).
\item Hindi pundit (B.A., Hindi from Dakshina bharat hindi prachar sabha).
\end{itemize}
\end{document}