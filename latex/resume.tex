\documentclass[11pt,a4paper]{article}
\usepackage{geometry}
\usepackage{titlesec}
\usepackage{ragged2e}
\usepackage[hidelinks]{hyperref}
\usepackage{tabto}
\usepackage{enumitem}
\usepackage{multicol}
\setlength{\parskip}{0em}
\hypersetup{
	colorlinks=false
}
\urlstyle{same}
%\titlespacing{\section}{0pt}{0em}{0pt}{}
\geometry{left=20mm,right=20mm,top=20mm,bottom=20mm}
\begin{document}
\begin{center}
{\LARGE{\textbf{ADHITHYAN V}}}\\
{Ph:9498090493\hfill}{\hfill v.adhithyan@gmail.com} \\
\emph{Site} \textbf{\href{https://v-adhithyan.github.io}{https://v-adhithyan.github.io}} \\
\emph{Github} \textbf{\href{https://github.com/v-adhithyan}{https://github.com/v-adhithyan}}\\
\emph{Experience: 3.5+ years}
\noindent\makebox[\linewidth]{\rule{\paperwidth}{0.4pt}}
\end{center}
%\begin{multicols}{2}
%\section*{VISION}
%\par To contribute to the profitability and sustainability of organization with my expertise.
%\section*{MISSION}
%\par Seeking a challenging role in an organization where I can contribute to the growth of organization as well as develop my skills.
%\end{multicols}
\section*{SUMMARY}
\begin{itemize}[noitemsep]
\item Hands on experience in developing and maintaining highly scalable distributed systems.
\item Have experience working in SOA, micro-services and webservices.
\item Have exposure to Kafka and SQS.
\item Have hands on knowledege on working in aws.
\item Have hands on experience in working with SQL databases such as MYSQL and Postgres and NOSQL stores such as Redis, Cassandra and DynamoDB
\end{itemize}
\section*{EXPERIENCE}
\emph{\textbf {Backend Engineer, HappyFox}}{\hfill Nov 2019 - Present}
\begin{itemize}[noitemsep]
\item Working on developing integrations, REST api's, addressing security issues, bug fixes and feature development for HappyFox Helpdesk.
\item Slowly transitioning into full stack developer.
\item Currently working on following technologies: Python, Django, ember
\end{itemize}
\emph{\textbf {Software Development Engineer, Fourkites}}{\hfill April 2018 - Oct 2018}
\begin{itemize}[noitemsep]
\item  Initiated the Fourkites Chatbot project and took ownership.Did market research and developed POC. Currently working on developing core functionalities of chatbot. My responsibilities include integrating different fourkites modules to chatbot, identifying and allocating tasks to my team mates.
\item My major work in Fourkites involves making the core platform stable by identifying architectural issues, areas for refactoring, reducing bug counts and making sure that only less bugs are created in the issue tracker.
\item Currently working on following technologies: Ruby on Rails, Postgres, Redis, Angular JS, Python Flask, DynamoDB and AWS. 
\end{itemize}
%\emph{Freelance Software Developer}{\hfill June 2016 -  March 2018}
%\begin{itemize}[noitemsep]
%\item Developed 4 android apps under my own banner.
%\item Did software development in upwork.
%\end{itemize}
\emph{\textbf {Member Technical Staff, Zoho}}{\hfill June 2015 - May 2016}
\begin{itemize}[noitemsep]
%\item Worked in Zoho Social and Zoho CRM.
%\item In Zoho CRM, I monitored Cassandra for a feature called Sales Inbox with a colleague.
%\item In Zoho Social, I fixed bugs, optimized code base, coded REST API's and proposed a new feature. \emph {Technologies used: Java, MySQL, Redis, Cassandra}
\item I became familiar with hg, issue tracker, post man and how to navigate through large code base and fixing bugs super fast.  
\item I adhered to strict SLA's and always finished tasks much before deadline, ensuring QA gets enough time to test. 
\item I was part of two teams Zoho Social and Zoho CRM. 
\item I developed a backend functionality that reduced API calls to Facebook Graph API by more than 40\%  that caches the permissions granted by user. This reduced graph API usage throttle significantly. 
\item I voluntarily fixed some of the bugs on my own that took more time to execute in server. I profiled the code that took more time in server and fixed those by viewing the source code of language and often implemented a version that is suitable for the app requirements. 
\item I became familiar with OAuth and have developed many REST API's here. 
\item I took ownership of some of the REST API modules and also developed few API's from scratch. 
\item I initiated revamping the analytics feature used in Zoho Social and did  market research and came up with POC. I was also responsible for maintaining and monitoring Cassandra. In addition to the job responsibilities, I also learnt many new technologies.
\end{itemize}
\emph{\textbf{Intern, Zoho}}{\hfill December 2014 - February 2015}
\begin{itemize}[noitemsep]
\item Studied code base of Zoho Social and became familiar with various internal frameworks, IDE used for development and testing in Zoho.
\end{itemize}
\section*{EDUCATION}
\textbf{MBA., (HR, Finance) {\hfill March 2018}}\\
College of Engineering, Guindy - Anna University {\hfill \textbf{CGPA: 8.8/10}}\\
\textbf{BE., Computer Science and Engineering {\hfill April 2015}}\\
College of Engineering, Guindy - Anna University {\hfill \textbf{CGPA: 8.03/10}}\\
\section*{SKILLS}
\TabPositions{3cm}
\begin{itemize}[noitemsep]
\item Languages \tab Ruby, Java, Kotlin, Swift, Python, C, C++, golang, Javascript, node.js, PHP
\item Frameworks \tab Ruby on Rails, Angular, Python Flask, Struts 2
\item Databases \tab MySQL, MongoDB, Cassandra, Redis, Dynamodb
\item IDEs, SDK \tab Netbeans, Eclipse, Android SDK, Xcode, Atom, Android Studio, Pycharm
\item Testing tools \tab Selenium
\item Others \tab Microsoft office, Git, Mercurial (hg), Photoshop, Jira, Github, Gitlab
\end{itemize}
%\newpage
%\section*{OPEN SOURCE PROJECTS}
%\emph{\textbf{Quper = Quote + Wallpaper (Android app)}}{\hfill Jan 2018}
%\begin{itemize}[noitemsep]
%\item Android app that is similar to whatsapp status feature with an option to save the created status and set as wallpaper. \emph{Technology used: Kotlin}
%\item \emph{Github} : \href{https://github.com/v-adhithyan/quper}{https://github.com/v-adhithyan/quper}
%\item Available on \href{https://play.google.com/store/apps/details?id=ceg.avtechlabs.quper}{\emph{Playstore}}.
%\end{itemize}
%\emph{\textbf{Drona (Android app)}}{\hfill May 2017}
%\begin{itemize}[noitemsep]
%\item Android app that displays management related news from hand curated list of various management sites. \emph{Technology used: Kotlin}
%\item \emph{Github} : \href{https://github.com/v-adhithyan/drona}{https://github.com/v-adhithyan/drona}
%\item Available on \href{https://play.google.com/store/apps/details?id=ceg.avtechlabs.mba}{\emph{Playstore}}.
%\end{itemize}
%\emph{\textbf{itunes-controller (Python + Opencv + Mac app)}}{\hfill August 2016}
%\begin{itemize}[noitemsep]
%\item A macOS background utility that recognizes person in front of system and controls itunes playback. \emph{Built using
%Python, OpenCV, Applescript.}
%\item It is available as a pip package.
%\item \emph{Github} : \href{https://github.com/v-adhithyan/itunes-controller}{https://github.com/v-adhithyan/itunes-controller}
%\end{itemize}
%\emph{\textbf{PrettySeekBar - Android library}}{\hfill June 2016}
%\begin{itemize}[noitemsep]
%\item Circular Seekbar library for Android inspired from analog clock. \emph{Technology used: Java.}
%\item \emph{Github} : \href{https://github.com/v-adhithyan/PrettySeekBar}{https://github.com/v-adhithyan/}
%\end{itemize}
%\emph{\textbf{AutoLogout - Chrome extension}}{\hfill February - March 2016}
%\begin{itemize}[noitemsep]
%\item A chrome extension that logs out from popular social networking sites after 10 minutes. \emph{Built using javascript.}
%\item Available in \href{https://chrome.google.com/webstore/detail/auto-logout/affkccgnaoeohjnojjnpdalhpjhdiebh?hl=en}{\emph{Chrome web store}}.
%\item \emph{Github} : \href{https://github.com/v-adhithyan/AutoLogout}{https://github.com/v-adhithyan/AutoLogout}
%\end{itemize}
%\emph{\textbf{Cricket Score Applet - Mac app}}{\hfill March 2016}
%\begin{itemize}[noitemsep]
%\item A macOS menulet that displays live cricket score and news. \emph{Built using: swift.}
%\item \emph{Github} : \href{https://github.com/v-adhithyan/cricket-score-applet}{https://github.com/v-adhithyan/cricket-score-applet}
%\end{itemize}
%\emph{Research paper trending topic finder}{\hfill December 2014 - April 2015}
%\begin{itemize}[noitemsep]
%\item  IEEE-KDE dataset was used to find the trending areas of research for years 2012, 2013 and 2014. \emph{Technologies
%used: Java, Hadoop, MySQL.}
%\end{itemize}
%\emph{Comprehension answering system}{\hfill March 2014}
%\begin{itemize}[noitemsep]
%\item Developed as a part of Compilers lab. \emph{Technologies used: flex, yacc.}
%\item When a passage and a question is given, the system will display relevant answer from the passage.
%\end{itemize}
%\emph{Offmail}{\hfill March 2014}
%\begin{itemize}[noitemsep]
%\item Developed as a part of FOSS lab. \emph{Technologies used: PHP, MySQL.}
%\item The main functionality of system is the composed mail will be saved when the system is offline and it will be sent when internet
%becomes available.
%\end{itemize}
%\emph{Intelligent contact finder}{\hfill March 2014}
%\begin{itemize}[noitemsep]
%\item Developed as a part of Mobile lab. \emph{Technologies used: Android, Java, SQLite.}
%\item Android app that list contacts based on descending frequencies of calls made when searched.
%\end{itemize}
%\emph{TextEditor}{\hfill March 2013}
%\begin{itemize}[noitemsep]
%\item Developed as a part of Java lab. \emph{Built using: Java}.
%\item A simple text editor which can be used to open and save text files. Additionally users can find and replace
%words.
%\end{itemize}
%\section* {PUBLICATIONS}
%\begin{multicols}{2}
%\begin{itemize}[noitemsep]
%\item \href{http://www.ijstm.com/images/short_pdf/1427308779_1228.pdf}{IJSTM - A Novel Ontology based Framework for mapping Research Paper Titles to topic domains}
%\item \href{https://www.ijser.org/researchpaper/A-Novel-Ontology-based-Framework-for-Trend-Detection-from-Research-Paper-Titles.pdf}{IJSER - A Novel Ontology based Framework for Trend Detection from Research Paper Titles}
%\end{itemize}
%\end{multicols}
%\section*{MISC}

%\begin{itemize}[noitemsep]
%\item Completed online machine learning course offered by Andrew NG in coursera.org with 93.2\%.
%\item Organized debugging event of aBACUS 2015, an inter college symposium conducted by CEG.
%\item Secured educational district 3rd place in 12th Board Exams (2010).
%\item Hindi pundit (B.A., Hindi from Dakshina bharat hindi prachar sabha).
%\end{itemize}

\end{document}
